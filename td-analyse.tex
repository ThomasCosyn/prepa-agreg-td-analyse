\documentclass[12pt]{article}
\usepackage[utf8]{inputenc}
\usepackage[T1]{fontenc}
\usepackage[french]{babel}
\usepackage{amsmath, amssymb, amsthm}
\usepackage{enumitem}
\usepackage{geometry}
\geometry{a4paper, margin=1.5cm}


\title{\Large \textbf{TD Analyse}}
\author{Thomas Goossaert-Cosyn}

\newtheorem{exercise}{Exercice}

% Alignement des puces à gauche et espacement réduit
\setlist[itemize]{leftmargin=*, itemsep=0.3em}



\begin{document}
	\maketitle
	
	\setcounter{section}{1}
	\section{Fonctions d'une variable réelle}
	
	\setcounter{exercise}{23}
	\begin{exercise}
		Soit $f$ une fonction réelle définie sur un intervalle ouvert $I$. On suppose que pour tout $x \in I$ on a
		\[
		\liminf_{h \to 0^+} \frac{f(x+h) + f(x-h) - 2f(x)}{2h} > 0.
		\]
		Montrer que $f$ est convexe sur $I$.
		
		Le résultat reste vrai si on suppose seulement que la $\liminf$ est $\geq 0$. Voyez-vous comment faire ?
	\end{exercise}
	
	\begin{proof}
		Posons les fonctions pentes
		\[s_+(x,h)=\frac{f(x+h)-f(x)}{h}, s_-(x,h)=\frac{f(x)-f(x-h)}{h}\]
		L'hypothèse de l'énoncé se réécrit alors
		\[\forall x\in I, \liminf_{h\rightarrow 0^{+}} s_+(x,h)-s_-(x,h) > 0\]
		Pour passer de cette propriété locale à une inégalité globale, considérons $[a,c]\subset I$ et $[x_0=a,...,x_n=c]$ une subdivision de pas $\Delta=\frac{c-a}{n}$.\\
		On définit la suite de pentes locales par :
		\[s_k=\frac{f(x_{k+1})-f(x_k)}{\Delta}\]
		L'hypothèse donne don que $s_0<s_1<...<s_{n-1}$\\
		Soit $b\in [a,c]$, $n$ suffisamment grand et $k$ tel que $x_k\leq b<x_{k+1}$, alors la croissance des pentes entraine
		\[\frac{f(b)-f(a)}{b-a}\leq s_k<\frac{f(c)-f(b)}{c-b}\]
		D'où la convexité de $f$.
	\end{proof}
	
	\setcounter{section}{3}
	\section{Topologie}
	
	\setcounter{exercise}{2}
	\begin{exercise}
		Soit $K, L$ deux espaces métriques compact et $\pi : K \rightarrow L$ une bijection continue.\\
		Montrer que $\pi$ est un homéomorphisme entre $K$ et $L$
	\end{exercise}
	\begin{proof}
		Pour prouver que $\pi$ est un homéomorphisme, il suffit de montrer que sa réciproque $\pi^{-1} : L \rightarrow K$ est continue.\\
		
		Soit $(y_n)_{n \in \mathbb{N}}$ une suite d'éléments de $L$ convergeant vers un élément $y \in L$. Posons :
		\[ x_n = \pi^{-1}(y_n) \quad \text{et} \quad x = \pi^{-1}(y) \]
		Montrons que la suite $(x_n)$ converge vers $x$ dans $K$.
		
		La suite $(x_n)$ est une suite d'éléments de l'espace métrique compact $K$. 
		Par Bolzano-Weierstrass, de toute suite d'un compact, on peut extraire une sous-suite convergente. Soit $(x_{\phi(n)})$ une telle sous-suite, et notons $z \in K$ sa limite :
		\[ x_{\phi(n)} \xrightarrow[n \to \infty]{} z \]
		
		L'application $\pi$ est continue sur $K$. Par caractérisation séquentielle de la continuité, l'image de la sous-suite doit converger vers l'image de sa limite :
		\[ \pi(x_{\phi(n)}) \xrightarrow[n \to \infty]{} \pi(z) \]
		Or, par construction, $\pi(x_{\phi(n)}) = y_{\phi(n)}$. 
		
		Comme la suite parente $(y_n)$ converge vers $y$, toute sous-suite extraite converge également vers $y$. On a donc, par unicité de la limite dans $L$ :
		\[ \pi(z) = y \]
		Puisque $\pi$ est une bijection, chaque élément possède un unique antécédent. Sachant que $\pi(x) = y$, l'injectivité de $\pi$ impose :
		\[ z = x \]
		
		Nous avons montré que toute sous-suite convergente de $(x_n)$ converge nécessairement vers $x$. Dans un espace compact, si une suite possède une unique valeur d'adhérence, alors elle converge vers cette valeur.
		
		Ainsi, $x_n \xrightarrow[n \to \infty]{} x$, ce qui prouve que $\pi^{-1}$ est continue en tout point $y \in L$. 
		
		Conclusion : $\pi$ est une bijection continue dont la réciproque est continue, c'est donc un homéomorphisme.
	\end{proof}
	

	\section{Intégration}
	
	\setcounter{exercise}{3}
	\begin{exercise}
		On travaille avec la mesure de Lebesgue sur $\mathbb{R}$ et on se donne une fonction continue par morceaux (par exemple) sur $\mathbb{R}$.
		
		\textbf{(1)} Montrer que si $f$ est positive, on a, lorsque $R \to +\infty$,
		\[
		\int_{-R}^{R} f \longrightarrow \int_{\mathbb{R}} f .
		\]
		
		\textbf{(2)} Montrer que si $f$ est intégrable, on a, lorsque $R \to +\infty$,
		\[
		\int_{-R}^{R} f \longrightarrow \int_{\mathbb{R}} f .
		\]
		
		\textbf{(3)} On suppose $f$ intégrable sur $\mathbb{R}$. Montrer que pour tout $\varepsilon > 0$, on peut trouver $g$ intégrable, bornée et à support compact telle que
		\[
		\int_{\mathbb{R}} |f-g| \le \varepsilon .
		\]
	\end{exercise}
		
		\begin{proof}
			\textbf{(1)}
			Supposons $f$ mesurable positive. Pour $R>0$, posons
			\[
			f_R = f \mathbf{1}_{[-R,R]}.
			\]
			Alors $(f_R)_{R>0}$ est une famille croissante de fonctions mesurables positives et
			\[
			f_R(x) \uparrow f(x) \quad \text{pour tout } x \in \mathbb{R}.
			\]
			Par le théorème de convergence monotone,
			\[
			\lim_{R \to +\infty} \int_{\mathbb{R}} f_R
			=
			\int_{\mathbb{R}} \lim_{R\to+\infty} f_R
			=
			\int_{\mathbb{R}} f.
			\]
			Or
			\[
			\int_{\mathbb{R}} f_R
			=
			\int_{-R}^{R} f.
			\]
			On en déduit
			\[
			\int_{-R}^{R} f \longrightarrow \int_{\mathbb{R}} f.
			\]
			
			\textbf{(2)} Supposons maintenant $f \in L^1(\mathbb{R})$. Alors $|f|$ est intégrable. On applique le résultat précédent à la fonction positive $|f|$ :
			\[
			\int_{-R}^{R} |f| \longrightarrow \int_{\mathbb{R}} |f|.
			\]
			On écrit
			\[
			\int_{\mathbb{R}} f - \int_{-R}^{R} f
			=
			\int_{|x|>R} f(x)\,dx.
			\]
			Donc
			\[
			\left| \int_{\mathbb{R}} f - \int_{-R}^{R} f \right|
			\le
			\int_{|x|>R} |f|.
			\]
			Or
			\[
			\int_{|x|>R} |f|
			=
			\int_{\mathbb{R}} |f| - \int_{-R}^{R} |f|
			\longrightarrow 0.
			\]
			Ainsi
			\[
			\int_{-R}^{R} f \longrightarrow \int_{\mathbb{R}} f.
			\]
			
			\textbf{(3)} Soit $f \in L^1(\mathbb{R})$ et $\varepsilon > 0$.
			
			
			\textbf{Étape 1 : troncature du support.}
			
			D'après le point (2), il existe $R>0$ tel que
			\[
			\int_{|x|>R} |f(x)|\,dx < \frac{\varepsilon}{2}.
			\]
			Posons
			\[
			f_1 = f \mathbf{1}_{[-R,R]}.
			\]
			Alors
			\[
			\int_{\mathbb{R}} |f-f_1|
			=
			\int_{|x|>R} |f|
			<
			\frac{\varepsilon}{2}.
			\]
			
			
			\textbf{Étape 2 : troncature des valeurs.}
			
			Pour $M>0$, posons
			\[
			f_{1,M}(x) =
			\begin{cases}
				f_1(x) & \text{si } |f_1(x)| \le M, \\
				M\,\mathrm{sgn}(f_1(x)) & \text{sinon}.
			\end{cases}
			\]
			Alors $f_{1,M}$ est bornée par $M$ et a toujours son support dans $[-R,R]$.
			
			Comme $f_1 \in L^1$, on a
			\[
			\int_{\{|f_1|>M\}} |f_1| \longrightarrow 0
			\quad \text{lorsque } M \to +\infty.
			\]
			Donc il existe $M>0$ tel que
			\[
			\int_{\{|f_1|>M\}} |f_1|
			<
			\frac{\varepsilon}{2}.
			\]
			Or
			\[
			|f_1 - f_{1,M}|
			\le
			|f_1| \mathbf{1}_{\{|f_1|>M\}},
			\]
			donc
			\[
			\int_{\mathbb{R}} |f_1 - f_{1,M}|
			<
			\frac{\varepsilon}{2}.
			\]
			
			\medskip
			
			\textbf{Conclusion.}
			
			Posons $g = f_{1,M}$. Alors $g$ est intégrable, bornée, à support compact et
			\[
			\int_{\mathbb{R}} |f-g|
			\le
			\int_{\mathbb{R}} |f-f_1|
			+
			\int_{\mathbb{R}} |f_1-g|
			<
			\frac{\varepsilon}{2}
			+
			\frac{\varepsilon}{2}
			=
			\varepsilon.
			\]
			
			\bigskip
			
			Ceci achève la démonstration.
		\end{proof}
	
	\setcounter{section}{7}
	\section{Transformée de Fourier}
	
	\setcounter{exercise}{4}
	\begin{exercise}
		On travaille dans l'espace de Hilbert $H = L^2(\mathbb{R}, \mu)$ associé à la mesure (gaussienne standard) de probabilité $\mu$ sur $\mathbb{R}$ définie par $d\mu(x) = \frac{e^{-x^2/2}}{\sqrt{2\pi}}dx$. Pour $n \in \mathbb{N}$ et $x \in \mathbb{R}$, on note :
		\[ h_n(x) = e^{x^2/2}g^{(n)}(x), \quad \text{où} \quad g(x) = e^{-x^2/2}. \]
		
		\begin{enumerate}
			\item Montrer que les fonctions $h_n$ sont polynomiales. Déterminer leurs degrés et coefficients dominants.
			\item Montrer que $(h_n)_{n\in\mathbb{N}}$ forme une famille orthogonale de $H$ et calculer la norme $\lambda_n$ de chaque élément $h_n$.
			\item On veut montrer que la famille $(\lambda_n^{-1}h_n)_{n\in\mathbb{N}}$ est une base hilbertienne de $H$.
			\begin{enumerate}
				\item Expliquer pourquoi il suffit de montrer que les polynômes sont denses dans $H$.
				\item Pour $f \in H$ et $x \in \mathbb{R}$, on note $\psi_f(x) = f(x)e^{-x^2/2}$. Montrer que sa transformée de Fourier $\widehat{\psi_f}$ est bien définie sur $\mathbb{R}$ et qu'elle s'étend en une fonction holomorphe sur $\mathbb{C}$.
				\item Pour tout $n \in \mathbb{N}$, calculer la dérivée $n$-ième $\widehat{\psi_f}^{(n)}(0)$. En déduire que si $f$ est orthogonale à tous les polynômes, alors $f$ est nulle.
				\item Conclure.
			\end{enumerate}
		\end{enumerate}
	\end{exercise}
	
	\begin{proof}
		1. Par récurrence, montrons que $g^{(n)}(x) = P_n(x)e^{-x^2/2}$ où $P_n$ est un polynôme de degré $n$ et de coefficient dominant $(-1)^n$.
		\begin{itemize}
			\item Pour $n=0$, $g^{(0)}(x) = e^{-x^2/2}$, donc $P_0(x)=1$. La propriété est vraie.
			\item Supposons $g^{(n)}(x) = P_n(x)e^{-x^2/2}$. Alors :
			\[ g^{(n+1)}(x) = \frac{d}{dx}(P_n(x)e^{-x^2/2}) = (P_n'(x) - x P_n(x))e^{-x^2/2} \]
			Ainsi $P_{n+1}(x) = P_n'(x) - x P_n(x)$. Si $P_n$ est de degré $n$ et de coefficient dominant $a_n$, alors $-xP_n$ est de degré $n+1$ et de coefficient dominant $-a_n$. Le terme $P_n'$ est de degré $n-1$, il ne modifie pas le degré dominant.
		\end{itemize}
		Par construction, $h_n(x) = e^{x^2/2} (P_n(x)e^{-x^2/2}) = P_n(x)$. 
		\textbf{Conclusion :} $h_n$ est un polynôme de \textbf{degré $n$} et de \textbf{coefficient dominant $(-1)^n$}.\\
		
		2. Soient $m, n \in \mathbb{N}$ avec $m < n$. Le produit scalaire dans $H$ est :
		\[ \langle h_n, h_m \rangle_\mu = \int_{\mathbb{R}} h_n(x) h_m(x) \frac{e^{-x^2/2}}{\sqrt{2\pi}} dx = \frac{1}{\sqrt{2\pi}} \int_{\mathbb{R}} \left( e^{x^2/2} g^{(n)}(x) \right) h_m(x) e^{-x^2/2} dx = \frac{1}{\sqrt{2\pi}} \int_{\mathbb{R}} g^{(n)}(x) h_m(x) dx \]
		Par intégrations par parties successives ($n$ fois), les termes de bord s'annulent car les dérivées de la gaussienne tendent vers $0$ :
		\[ \langle h_n, h_m \rangle_\mu = \frac{(-1)^n}{\sqrt{2\pi}} \int_{\mathbb{R}} g(x) h_m^{(n)}(x) dx \]
		Comme $\deg(h_m) = m < n$, on a $h_m^{(n)}(x) = 0$, donc $\langle h_n, h_m \rangle_\mu = 0$. La famille est orthogonale.
		
		Pour la norme $\lambda_n^2 = \langle h_n, h_n \rangle_\mu$ :
		\[ \lambda_n^2 = \frac{(-1)^n}{\sqrt{2\pi}} \int_{\mathbb{R}} g(x) h_n^{(n)}(x) dx \]
		Or $h_n(x) = (-1)^n x^n + \dots$, donc $h_n^{(n)}(x) = (-1)^n n!$. D'où :
		\[ \lambda_n^2 = \frac{(-1)^n (-1)^n n!}{\sqrt{2\pi}} \int_{\mathbb{R}} e^{-x^2/2} dx = \frac{n!}{\sqrt{2\pi}} \times \sqrt{2\pi} = n! \]
		\textbf{Conclusion :} $\lambda_n = \sqrt{n!}$.\\
		
		3.a) Puisque $\deg(h_n) = n$, toute combinaison linéaire des $h_k$ ($k \le n$) décrit l'espace des polynômes de degré au plus $n$. Si les polynômes sont denses dans $H$, alors l'adhérence de $\text{Vect}(h_n)_{n\in\mathbb{N}}$ est $H$. Comme la famille est orthonormale (après division par $\lambda_n$), c'est une base hilbertienne.\\
		
		3.b) $f \in H$ signifie $\int |f|^2 d\mu < \infty$. $\psi_f(x) = f(x)e^{-x^2/2}$.\\
		Par Cauchy-Schwarz, sa transformée de Fourier $\widehat{\psi_f}(\xi) = \int f(x) e^{-x^2/2} e^{-ix\xi} dx$ est bien définie car $\psi_f \in L^1(\mathbb{R})$ :
		\[ \int |f(x)| e^{-x^2/2} dx = \sqrt{2\pi} \int |f(x)| e^{-x^2/4} \frac{e^{-x^2/4}}{\sqrt{2\pi}} dx \le \sqrt{2\pi} \|f\|_H \|e^{-x^2/4}\|_H < \infty \]
		L'extension holomorphe $F(z) = \int f(x) e^{-x^2/2} e^{-ixz} dx$ pour $z \in \mathbb{C}$ est justifiée par le fait que pour $z = u+iv$, $|e^{-ixz}| = e^{xv}$. Le terme $e^{-x^2/2}$ assure la convergence et la domination pour tout $z \in \mathbb{C}$ (croissance exponentielle vs décroissance gaussienne).\\
		
		3.c) On a 
		\[\widehat{\psi_f}^{(n)}(0) = \int f(x) e^{-x^2/2} (-ix)^n dx = (-i)^n \sqrt{2\pi} \int f(x) x^n \frac{e^{-x^2/2}}{\sqrt{2\pi}} dx = (-i)^n \sqrt{2\pi} \langle f, x^n \rangle_\mu\]
		Si $f$ est orthogonale à tous les polynômes, alors $\langle f, x^n \rangle_\mu = 0$ pour tout $n$, donc $\widehat{\psi_f}^{(n)}(0) = 0$.
		Comme $\widehat{\psi_f}$ est une fonction analytique (holomorphe sur $\mathbb{C}$), ses coefficients de Taylor en 0 sont tous nuls, donc $\widehat{\psi_f} \equiv 0$. Par injectivité de la transformée de Fourier, $\psi_f = 0$ p.p., donc $f = 0$ p.p.\\
		
		3.d) L'orthogonal de l'espace des polynômes est $\{0\}$, donc les polynômes sont denses dans $H$. La famille $(\lambda_n^{-1} h_n)$ est donc une base hilbertienne.
	\end{proof}
	
	\setcounter{section}{9}
	\section{Equations différentielles}
	\setcounter{exercise}{2}
	\begin{exercise}
		Résoudre le problème différentiel suivant sur $\mathbb{R}$\\
		\[2x²y'=1-y²\]
	\end{exercise}
	
	\begin{proof}
		\begin{itemize}
			\item On travaille sur $I\subset \mathbb{R}$. Posons $f(x,y)=\frac{1-y²}{2x²}$. $f$ est :\\
			\begin{itemize}
				\item continue sur $I\times \mathbb{R}$
				\item $\mathcal{C}¹$ en $y$ donc localement lipschitzienne en $y$
			\end{itemize}
			Par le théorème de Cauchy Lipschitz, pour tout $(x_0,y_0)$, il existe une unique solution définie sur $I$.
		\end{itemize}
	\end{proof}
	
	\setcounter{exercise}{5}
	\begin{exercise}
		\textbf{Exercice 6.} Soit une fonction continue $A : \mathbb{R} \to M_n(\mathbb{R})$. On note $(y_1,\dots,y_n)$ une base de solutions de l’équation différentielle
		\[
		y'(t)=A(t)y(t), \qquad t\in\mathbb{R}.
		\]
		On notera aussi $M(t)$ la matrice $(y_1(t),\dots,y_n(t))$ dans la base canonique de $\mathbb{R}^n$.
		\begin{enumerate}
			\item Calculer le wronskien $W(t)=\det M(t)$, pour tout $t\in\mathbb{R}$.
			\item On suppose maintenant que $\|A(\cdot)\|$ est intégrable sur $\mathbb{R}_+$.
			\begin{enumerate}
				\item Montrer que toute solution $y$ admet une limite en $+\infty$.
				\item Soit $\theta$ l’application qui à $y_0\in\mathbb{R}^n$ associe la limite en $+\infty$ de la solution $y$ telle que $y(0)=y_0$. Montrer que $\theta$ est un isomorphisme de $\mathbb{R}^n$ sur $\mathbb{R}^n$.
			\end{enumerate}
		\end{enumerate}
		\end{exercise}
		
		\begin{proof}
		
		
		1. La matrice fondamentale $M(t)$ vérifie
		\[
		M'(t)=A(t)M(t).
		\]
		On pose $W(t)=\det M(t)$. Par la formule de dérivation du déterminant,
		\[
		W'(t)=\operatorname{tr}(A(t))\,W(t).
		\]
		On obtient donc une équation différentielle scalaire :
		\[
		W'(t)=\operatorname{tr}(A(t))\,W(t).
		\]
		En intégrant,
		\[
		W(t)=W(0)\exp\!\left(\int_0^t \operatorname{tr}(A(s))\,ds\right).
		\]
		En particulier, si $(y_1,\dots,y_n)$ est une base de solutions, alors $W(0)\neq 0$, donc
		\[
		W(t)\neq 0 \quad \text{pour tout } t\in\mathbb{R}.
		\]
		
		2.a) Soit $y$ une solution. Elle vérifie
		\[
		y(t)=y(0)+\int_0^t A(s)y(s)\,ds.
		\]
		Pour $t\ge s\ge 0$,
		\[
		y(t)-y(s)=\int_s^t A(\tau)y(\tau)\,d\tau.
		\]
		On en déduit
		\[
		\|y(t)-y(s)\|
		\le
		\int_s^t \|A(\tau)\|\,\|y(\tau)\|\,d\tau.
		\]
		Par ailleurs on a également,
		\[\|y(t)\|\leq \|y(0)\| + \int_{0}^{t}\|A(s)y(s)\|ds\]
		Donc en appliquant le lemme de Gronwall pour $\|A(\cdot)\|$ positive et $\|y(\cdot)\|$ continue, pour $t\ge 0$,
		\[
		\|y(t)\|
		\le
		\|y(0)\|\exp\!\left(\int_0^t \|A(\sigma)\|\,d\sigma\right).
		\]
		Comme $\|A(\cdot)\|$ est intégrable sur $\mathbb{R}_+$, la quantité
		\[
		C=\exp\!\left(\int_0^{+\infty} \|A(\sigma)\|\,d\sigma\right)
		\]
		est finie, et donc
		\[
		\|y(t)\|\le C\|y(0)\| \quad \text{pour tout } t\ge 0.
		\]
		Ainsi $y$ est bornée sur $\mathbb{R}_+$.
		
		Dès lors,
		\[
		\|y(t)-y(s)\|
		\le
		C\|y(0)\|\int_s^t \|A(\tau)\|\,d\tau.
		\]
		Comme $\|A\|$ est intégrable sur $\mathbb{R}_+$, le membre de droite tend vers $0$ lorsque $s,t\to+\infty$.  
		
		Donc $y(t)$ est de Cauchy lorsque $t\to+\infty$, et comme $\mathbb{R}^n$ est complet, $y(t)$ admet une limite lorsque $t\to+\infty$.\\
		
		2.b) On définit
		\[
		\theta : \mathbb{R}^n \to \mathbb{R}^n,
		\qquad
		\theta(y_0)=\lim_{t\to+\infty} y(t),
		\]
		où $y$ est l’unique solution telle que $y(0)=y_0$.
		
		D'après la question précédente, $\theta$ est bien définie. Elle est linéaire car l’équation différentielle est linéaire.
		
		Montrons qu’elle est injective.  
		
		Si $\theta(y_0)=0$, alors $y(t)\to 0$ lorsque $t\to+\infty$.  
		On écrit, pour $t\ge 0$,
		\[
		y(t)=y(T)-\int_t^T A(s)y(s)\,ds.
		\]
		En faisant tendre $T\to+\infty$ et en utilisant que $y(T)\to 0$, on obtient
		\[
		y(t)=-\int_t^{+\infty} A(s)y(s)\,ds.
		\]
		On en déduit
		\[
		\|y(t)\|
		\le
		\int_t^{+\infty} \|A(s)\|\,\|y(s)\|\,ds.
		\]
		En posant
		\[
		m(t)=\sup_{u\ge t}\|y(u)\|,
		\]
		qui est bien défini car $\|y(\cdot)\|$ est continue et $y(t)\rightarrow 0$ en $+\infty$, on obtient
		\[
		m(t)\le m(t)\int_t^{+\infty}\|A(s)\|\,ds.
		\]
		Et comme $\|A(\cdot)\|$ est intégrable, on a pour $t$ assez grand,
		\[
		\int_t^{+\infty}\|A(s)\|\,ds<1,
		\]
		donc nécessairement $m(t)=0$. Ainsi $y(t)=0$ pour $t$ assez grand, puis par unicité des solutions, $y\equiv 0$. Donc $y_0=0$.
		
		Ainsi $\theta$ est injective.  
		
		Comme $\theta$ est linéaire entre deux espaces vectoriels de même dimension finie $n$, elle est bijective.  
		
		Donc $\theta$ est un isomorphisme de $\mathbb{R}^n$ sur $\mathbb{R}^n$.
		
	
	\end{proof}
	
	
\end{document}
